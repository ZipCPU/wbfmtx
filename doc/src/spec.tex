%%%%%%%%%%%%%%%%%%%%%%%%%%%%%%%%%%%%%%%%%%%%%%%%%%%%%%%%%%%%%%%%%%%%%%%%%%%
%%
%% Filename: 	spec.tex
%%
%% Project:	Wishbone controlled FM Transmitter Hack
%%
%% Purpose:	This LaTeX file contains all of the documentation/description
%%		currently provided with this FM transmitter hack.  It's not
%%		nearly as interesting as the PDF file it creates, so I'd
%%		recommend reading that before diving into this file.  You
%%		should be able to find the PDF file in the SVN distribution
%%		together with this PDF file and a copy of the GPL-3.0 license
%%		this file is distributed under.  If not, just type 'make'
%%		in the doc directory and it (should) build without a problem.
%%
%%
%% Creator:	Dan Gisselquist
%%		Gisselquist Technology, LLC
%%
%%%%%%%%%%%%%%%%%%%%%%%%%%%%%%%%%%%%%%%%%%%%%%%%%%%%%%%%%%%%%%%%%%%%%%%%%%%
%%
%% Copyright (C) 2016, Gisselquist Technology, LLC
%%
%% This program is free software (firmware): you can redistribute it and/or
%% modify it under the terms of  the GNU General Public License as published
%% by the Free Software Foundation, either version 3 of the License, or (at
%% your option) any later version.
%%
%% This program is distributed in the hope that it will be useful, but WITHOUT
%% ANY WARRANTY; without even the implied warranty of MERCHANTIBILITY or
%% FITNESS FOR A PARTICULAR PURPOSE.  See the GNU General Public License
%% for more details.
%%
%% You should have received a copy of the GNU General Public License along
%% with this program.  (It's in the $(ROOT)/doc directory, run make with no
%% target there if the PDF file isn't present.)  If not, see
%% <http://www.gnu.org/licenses/> for a copy.
%%
%% License:	GPL, v3, as defined and found on www.gnu.org,
%%		http://www.gnu.org/licenses/gpl.html
%%
%%
%%%%%%%%%%%%%%%%%%%%%%%%%%%%%%%%%%%%%%%%%%%%%%%%%%%%%%%%%%%%%%%%%%%%%%%%%%%
\documentclass{gqtekspec}
\project{Wishbone Controlled FM Transmitter Hack}
\title{Specification}
\author{Dan Gisselquist, Ph.D.}
\email{dgisselq (at) opencores.org}
\revision{Rev.~0.1}
\begin{document}
\pagestyle{gqtekspecplain}
\titlepage
\begin{license}
Copyright (C) \theyear\today, Gisselquist Technology, LLC

This project is free software (firmware): you can redistribute it and/or
modify it under the terms of  the GNU General Public License as published
by the Free Software Foundation, either version 3 of the License, or (at
your option) any later version.

This program is distributed in the hope that it will be useful, but WITHOUT
ANY WARRANTY; without even the implied warranty of MERCHANTIBILITY or
FITNESS FOR A PARTICULAR PURPOSE.  See the GNU General Public License
for more details.

You should have received a copy of the GNU General Public License along
with this program.  If not, see \texttt{http://www.gnu.org/licenses/} for a
copy.
\end{license}
\begin{revisionhistory}
0.1 & 6/15/2016 & Gisselquist & First Draft \\\hline
\end{revisionhistory}
% Revision History
% Table of Contents, named Contents
\tableofcontents
% \listoffigures
\listoftables
\begin{preface}
After watching someone demonstrate a Python hack that turned a Raspberry Pi
into a poor man's FM transmitter, I decided that I should try to see if I could
do the same with an FPGA.  Indeed, it should be easier with an FPGA: the FPGA
has complete control of the clock, as well as what the data line does.
Therefore, this hack attempts to turn a GPIO line into an FM transmitter line
for an antenna.
\end{preface}

\chapter{Introduction}
\pagenumbering{arabic}
\setcounter{page}{1}

This project is a hack.

It is not intended, nor appropriate, for any commercial or otherwise
useful product.  Broadcasting on commercial FM channels has legal implications
associated with it.  I am not recommending that you turn your FPGA into an
illegal FM transmitter.

The purpose of this project is to show that an FPGA's outputs can be used to
create a (nearly) analog FM output.

As the preface mentions, this project is also about one-upsmanship.  Just
because your Raspberry Pi can do something doesn't mean my FPGA can't.  Here,
let me prove to you that an FPGA can create and broadcast on a commercial
FM radio channel.

As with any specification, this one is broken into sections or chapters.  
Chap.~\ref{ch:ops} will start off by explaining how to use this core.
Chap.~\ref{ch:regs} will then discuss the registers in detail.  This may seem
like rehashing the Chap.~\ref{ch:ops} chapter, but the information is presented
in a different order.  Chap.~\ref{ch:wb} then presents the wishbone data sheet
necessary for any wishbone compliant core.  Finally, Chap.~\ref{ch:io} walks
through the I/O ports of the core. 

% \chapter{Architecture}
\chapter{Operation}\label{ch:ops}

From a logical standpoint, the operation of this core is quite simple.  Just
follow the following steps:
\begin{enumerate}
\item Select the frequency ``channel'' to transmit on.
\item Adjust the sample rate to set how fast output samples will be sent to the
	device.
\item Send the first sample to the core
\item Wait for an interrupt, then send the next sample to the core
\item Repeat step 4 until the desired transmission is done.
\item Once transmission is complete, set the frequency ``channel'' slash 
	NCO step size to zero.
\item Set the next sample to zero.
\item Disable, in your interrupt controller (external to this core) the
	interrupt generated by this core.
\end{enumerate}

Internally, the core attempts to generate a square wave at a frequency given by
the set frequency plus an amount given by the sample value times a constant.
To do this, the core maintains a 32~bit counter which will roll over at the
carrier
frequency times per second.  The top bit of this counter becomes the output
bit for the transmitter.   The counter is incremented every clock by an amount
used to set the carrier frequency, plus an amount given by the input sample.

For example, let's assume that the FPGA is running with an 80~MHz clock.
To toggle the output line at a rate of 20~Mhz, one need only set the counter
increment to {\tt 0x40000000}.  The top bit will, over time, trace through
{\tt 0, 0, 1, 1 }--creating a square wave at 20~MHz.  As a rather interesting
by product of the fact that this is a square wave is that this 20~MHz tone
will have artifacts at odd harmonics of 20~MHz: 60~MHz, 100~MHz, 140~MHz, etc.
The energy in each of these harmonics will decrease, dependent upon both the
FPGA switching speed and the nature of a square wave.  In particular, the
100~MHz harmonic will have 13.6~dB less power than the fundamental at 20~MHz.

Now, if we add a value of {\tt 0x7fff} times 32 to this counter increment,
creating a new increment of {\tt 0x400fffe0}, the new counter will roll over
as many times in $2^{32}$ clocks, creating a frequency of roughly 
20.019~MHz. It's fifth harmonic, however, will be at 100.097~MHz, nicely at the
edge, if not a little beyond, the frequency range of FM broadcast radio.

By changing the offset to the counter increment with each sample, we create a
Frequency Modulation.  This is what allows us to generate an FM waveform
similar to that in the FM Broadcast band.

If only life were that simple, we'd be done at this point.

The next part of the operation of this hack is the antenna.  For best
performance, this output waveform needs to be fed into an antenna with a DC
block and ground as the other lead and a DC block.  Ideally, this antenna
should be impedence matched to the board as well.

For the purposes of our hack, we will ignore these details and hope to 
demonstrate success with just the previously discussed logic.

\section{Software Example}

Before leaving our concept of operation, let's walk through some code which 
was used to demonstrate this board.  The demonstration itself was done using
a ZipCPU, together with a modified version of the XuLA2-LX25 SoC, both
available from OpenCores.\footnote{That is, the XuLA2-LX25 SoC is available
from OpenCores, as is the ZipCPU, but the modified version is not posted.  It
just didn't seem worth it to maintain a simple hack there.}

The first step is to set the frequency channel of the board.  Here, we set it
to $91.9$~MHz, based upon an 80~MHz internal oscillator clock.
\begin{eqnarray}
{\mbox{\tt sys->io\_fmtx\_nco}} &=& \mbox{\tt 0x26147ae1;}
\end{eqnarray}

The next step is to set the sample rate of the device.  In my case, I set this
as a parameter to the module.  However, it can also be set here as a run time
configuration parameter:
\begin{eqnarray}
{\mbox{\tt sys->io\_fmtx\_audio}} &=& \mbox{\tt 1814<<16;}
\end{eqnarray}

For our example, we'll poll the interrupt controller to see when the
{\tt INT\_FM} interrupt line goes high:
\begin{eqnarray}
{\mbox{\tt while((sys->io\_pic \& INT\_FM)==0) ;}}
\end{eqnarray}

Once it goes high, we can send a sample to the transmitter,
\begin{eqnarray}
{\mbox{\tt sys->io\_fmtx\_audio}} &=& \mbox{\tt sample \& 0x0ffff};
\end{eqnarray}

We now repeat the process of checking the transmitter for readiness to send the next sample, and sending samples, until we are done.

Once complete, we simply turn the module off:
\begin{eqnarray}
{\mbox{\tt sys->io\_fmtx\_nco}} &=& \mbox{\tt 0;} \\
{\mbox{\tt sys->io\_fmtx\_audio}} &=& \mbox{\tt 0;}
\end{eqnarray}


That's it!  It's really quite simple to use.
 
\chapter{Registers}\label{ch:regs}

This FM Transmitter core supports two registers, as listed in
Tbl.~\ref{tbl:reglist}: a next sample register, {\tt SAMPLE}, and a carrier
frequency control register called {\tt NCOSTEP}.
\begin{table}[htbp]
\begin{center}
\begin{reglist}
SAMPLE	& 0 & 32 & R/W & Controls the sample value out of the transmitter,
	as well as the sample rate of the transmitters interrupts requesting
	further samples.\\\hline
NCOSTEP	& 1 & 32 & R(/W) & Controls the step size of the pseudo-oscillator
	controlling the RF frequency.  Appropriate writes to this register
	will determine what channel the FM transmitter broadcasts on.
        \\\hline
\end{reglist}\caption{List of Registers}\label{tbl:reglist}
\end{center}\end{table}
Each register will be discussed in detail in this chapter.

\section{Sample Register}
The bits in the control register are defined in Tbl.~\ref{tbl:sample}.
\begin{table}[htbp]
\begin{center}
\begin{bitlist}
16--31 & R/W &  This is the number of clocks between interrupts.  Hence, to
	transmit from a waveform file sampled at a rate of $R$~samples per
	second, from an FPGA with a clock rate of $F$~Hz, set this value
	to $F/R$.  For example, to transmit at 44.1~kHz from an FPGA with
	an 80~MHz clock, set this value to 1814.

	Writing a value of zero to this register has no effect, allowing
	a user to only write the sample value at each write without adjusting
	the sample rate.\\\hline
0--15 & W & Signed, twos complement, next sample to be broadcast.\\\hline
1--15 & R & Signed, twos complement, current sample being broadcast.\\\hline
0 & R & A 1'b1 if the interrupt is currently active, otherwise zero.  The
	actual lowest bit of the data value in the transmitter cannot be read
	out.\\\hline
\end{bitlist}
\caption{Sample Register}\label{tbl:sample}
\end{center}\end{table}

Basically, in sum, the top 16~bits determine the sample rate of the audio
being sent to the device.  Perhaps more accurately, they set the number of
clocks between assertions of the CPU interrupt line.  The core will internally
run a timer at an interval given by these bits.  When the timer is up, it will
transmit its next sample, assert an interrupt, and restart the timer with this
value.  The CPU will then have until the timer expires to provide the next
sample.  Writing to this register with these bits set to zero will cause them
to be ignored.

It should be possible to run this from a DMA controller, although I have not
tried to do so.

The lower 16 bits of this register, when written to, control the next audio
sample out of the device.  When read from, they return the current audio sample
being produced by the device, and in the low order bit whether or not an
interrupt is currently being asserted.

\section{Carrier Frequency Control Register}
Based upon Nyquist principles, properly producing a sampled tone requires 
samples that are at least twice the frequency of the desired tone.  In the
case of commercial FM in the US, the highest frequency may be roughly 110~MHz.
This means that the FPGA must produce a sampled output using a clock of at
least 220~MHz.  

My FPGA boards don't clock that high.  Instead, I can clock my Spartan--6 boards
at 80~MHz.  While this should be sufficient for transmitting in the Citizen's
Band of 26~to 28~MHz, it is entirely insufficient for transmitting at commercial
radio.

Instead, to reach these really high speeds, this core exploits what
is normally an undesired consequence of sampling: aliasing.  Basically, that 
means that it is possible to produce a tone at some frequency, such as 10~MHz,
as well as your clock rate plus that frequency, or 90~MHz in my case.  The 
90~MHz output is often considered an undesirable artifact of the square wave
outputs produced by the FPGA, but in our case we exploit this.

Now that all that is said, we can discuss setting the Carrier Frequency
Control Reigster.  This register is set when you wish to begin transmitting
to:
\begin{eqnarray}
{\tt CFCR} &=& \left\lfloor \frac{2^{32} f_{ch}}{f_{\mbox{\tiny FPGA}}}
		+\frac{1}{2}\right\rfloor
\end{eqnarray}
where $f_{ch}$ is the center frequency you wish to transmit on, and
$f_{\mbox{\tiny FPGA}}$ is your FPGA clock frequency.  Note that this value
will be greater than $2^{32}$ for my setup, since the frequency of my FPGA
is less than that of the channel I wish to transmit on.  In this case, just
throw away any bits above the lower thirty--two and continue.

As an example, my FPGA's clock runs at 80~MHz.  In order to transmit at 
91.9~MHz, I would then set the {\tt CFCR} register to 
\hbox{0x26147ae1}.

\chapter{Wishbone Datasheet}\label{chap:wishbone}\label{ch:wb}
Tbl.~\ref{tbl:wishbone}
\begin{table}[htbp]
\begin{center}
\begin{wishboneds}
Revision level of wishbone & WB B4 spec \\\hline
Type of interface & Slave, Read/Write, pipeline reads supported \\\hline
Port size & 32--bit \\\hline
Port granularity & 32--bit \\\hline
Maximum Operand Size & 32--bit \\\hline
Data transfer ordering & (Irrelevant) \\\hline
Clock constraints & None.\\\hline
Signal Names & \begin{tabular}{ll}
		Signal Name & Wishbone Equivalent \\\hline
		{\tt i\_wb\_clk} & {\tt CLK\_I} \\
		{\tt i\_wb\_cyc} & {\tt CYC\_I} \\
		{\tt i\_wb\_stb} & {\tt STB\_I} \\
		{\tt i\_wb\_we} & {\tt WE\_I} \\
		{\tt i\_wb\_addr} & {\tt ADR\_I} \\
		{\tt i\_wb\_data} & {\tt DAT\_I} \\
		{\tt o\_wb\_ack} & {\tt ACK\_O} \\
		{\tt o\_wb\_stall} & {\tt STALL\_O} \\
		{\tt o\_wb\_data} & {\tt DAT\_O}
		\end{tabular}\\\hline
\end{wishboneds}
\caption{Wishbone Datasheet}\label{tbl:wishbone}
\end{center}\end{table}
is required by the wishbone specification, and so 
it is included here.  The big thing to notice is that this core
acts as a wishbone slave, and that all accesses to any local
registers become 32--bit reads and writes to this interface. 

\chapter{IO Ports}\label{ch:io}

The ports are listed in Table.~\ref{tbl:ioports}.
\begin{table}[htbp]
\begin{center}
\begin{portlist}
{\tt i\_clk} & 1 & Input & The clock synchronizing the entire core.\\\hline
{\tt i\_wb\_cyc} & 1 & Input & Indicates a wishbone bus cycle is active when
		high.  \\\hline
{\tt i\_wb\_stb} & 1 & Input & Indicates a wishbone bus cycle for this
	peripheral when high.  (See the wishbone spec for more details) \\\hline
{\tt i\_wb\_we} & 1 & Input & Write enable, allows indicates a write to one of
	the two registers when {\tt i\_wb\_stb} is also high.
        \\\hline
{\tt i\_wb\_addr} & 1 & Input & A single address line, set to zero to access the
		configuration and control register, to one to access the data
		register.  \\\hline
{\tt i\_wb\_data} & 32 & Input & Data used when writing to the core.  Valid
		when {\tt i\_wb\_cyc}, {\tt i\_wb\_stb}, and {\tt i\_wb\_we}
		are all high, ignored otherwise.  \\\hline
{\tt o\_wb\_ack} & 1 & Output & Wishbone acknowledgement.  This line will go
		high on the clock after any wishbone access.\\\hline
{\tt o\_wb\_stall} & 1 & Output & Required by the wishbone spec, but always
		set to zero in this implementation.
		\\\hline
{\tt o\_wb\_data} & 32 & Output & Value read, whether the next sample register
	or the nco step register, headed back to the wishbone bus master. 
	These bits will be valid during any
        read cycle when the {\tt o\_wb\_ack} line is high.
        \\\hline
{\tt o\_tx} & 1 & Output & A one wire output value to be sent to the ``antenna''
	output pin of your FPGA.\\\hline
{\tt o\_int} & 1 & Output & True whenever the next sample has transitioned
	to the current sample, until a new next sample is written. \\\hline
\end{portlist}
\caption{List of IO ports}\label{tbl:ioports}
\end{center}\end{table}
Of these ports, the {\tt i\_wb\_*} and the {\tt o\_wb\_*} ports are all
defined by the wishbone specification.  This leaves two ports of interest,
{\tt o\_tx} and {\tt o\_int}.

The {\tt o\_tx} output is the FM transmitter output.  This output needs to be
wired off of your board to your FM transmit antenna.  Should your board not have
such an antenna, one can often be improvised by sending this output to any
available output ports from your FPGA.  The more GPIO's that are set with this
value, the more power the device will output and likewise the better the output
may approximate an FM antenna.

Finally, the {\tt o\_int} line is an interrupt line to be sent to whatever
controller is controlling the transmitter.  This interrupt line will be set
whenever the transmitter is ready for a new sample.  It is also self clearing, 
so that sending a sample to the transmitter will turn this off until the next
value is needed.

% Appendices
% Index
\end{document}


